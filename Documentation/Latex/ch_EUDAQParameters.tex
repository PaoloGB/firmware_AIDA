\chapter{EUDAQ Parameters}\label{ch:EUDAQPar}
List of parameters that are parsed by the EUDAQ run control \gls{gui} to configure the \gls{tlu}.\\
The parameters must be included in the INI or CONF file passed to the main window (see~fig.\ref{fig:EUDAQGui}).
\begin{figure}
  \centering
  \includegraphics[width=.90\textwidth]{./Images/RunControlGUI.jpg}
  \caption{Main user iterface of the EUDAQ framework.}
  \label{fig:EUDAQGui}
\end{figure}\\
Not all parameters are needed; if one of the parameters is not present in the files, the code will generally assume a default value, indicated in brackets in the following document \verb|[type, default]|.
\begin{description}
  \item[initid] \verb|[string, "0"]| Does not serve any purpose in the code but can be useful to identify configuration settings used in a specific run. EUDAQ will store this information in the run data.
  \item[ConnectionFile] \verb|[string, "file://./FMCTLU_connections.xml"]| Name of the xml file used to store the information required to communicate with the hardware, such as its IP address and the location of the address map. The default location indicates a file that must be located in the \texttt{bin} folder.
  \item[DeviceName] \verb|[string, "fmctlu.udp"]| The name of the type of hardware to be contacted by the IPBus.
  \item[TLUmod] \verb|[string, "1e"]| Version of the \gls{tlu} hardware. Reserved for future use.
  \item[nDUTs] \verb|[positive int, 4]| Number of \gls{dut} in the current \gls{tlu}. This is for future upgrades and should not require editing by the user.
  \item[nTrgIn] \verb|[positive int, 6]| Number of trigger inputs in the current \gls{tlu}. This is for future upgrades and should not require editing by the user.
  \item[I2C\_COREEXP\_Addr] \verb|[positive int, 0x21]| \gls{i2c} address of the core expander mounted on the Enclustra board. This is not required if a different \gls{fpga} is used.
  \item[I2C\_CLK\_Addr] \verb|[positive int, 0x68]| \gls{i2c} address of Si5345 clock generator installed on the \gls{tlu}.
  \item[I2C\_DAC1\_Addr] \verb|[positive int, 0x13]| \gls{i2c} address of \gls{dac} installed on the \gls{tlu}. The \gls{dac} is used to configure the threshold of the trigger inputs.
  \item[I2C\_DAC2\_Addr] \verb|[positive int, 0x1F]| \gls{i2c} address of \gls{dac} installed on the \gls{tlu}. The \gls{dac} is used to configure the threshold of the trigger inputs.
  \item[I2C\_ID\_Addr] \verb|[positive int, 0x50]| \gls{i2c} address the unique ID \gls{eeprom} installed on the \gls{tlu}. The chip is used to provide a unique identifier to each kit.
  \item[I2C\_EXP1\_Addr] \verb|[positive int, 0x74]| \gls{i2c} address the bus expander used to select the direction of the \gls{hdmi} pins on the board.
  \item[I2C\_EXP2\_Addr] \verb|[positive int, 0x75]| \gls{i2c} address the bus expander used to select the direction of the \gls{hdmi} pins on the board.
  \item[intRefOn] \verb|[boolean, false]| If true, the \gls{dac}s installed on the \gls{tlu} will use their internal voltage reference rather than the one provide externally.
  \item[VRefInt] \verb|[float, 2.5]| Value in volts for the internal reference voltage of the \gls{dac}s. The voltage is chosen by the chip manufacturer. This is only used if \verb|intRefOn= true|.
  \item[VRefExt] \verb|[float, 1.3]| Value in volts for the external reference voltage of the \gls{dac}s. The voltage is determined by a circuit on the \gls{tlu} and the value of this parameter must reflect such voltage. This is only used if \verb|intRefOn= false|.
  \item[CONFCLOCK] \verb|[bool, true]| If true, the clock chip Si5345 will be re-configured when the INIT button is pressed (see figure~fig.\ref{fig:EUDAQGui}). The chip is configured via \gls{i2c} interface using a specific text file (see next parameter). After a power cycle, the chip is not configured and must be reconfigured to operate the \gls{tlu} correctly.
  \item[CLOCK\_CFG\_FILE] \verb|[string, "./../user/eudet/misc/fmctlu_clock_config.txt"]| Name of the text file used to store the configuration values of the Si5345. The file can be generate using the Clockbuilder Pro software provided by \href{https://www.silabs.com/products/development-tools/software/clock}{SiLabs}.
\end{description}
