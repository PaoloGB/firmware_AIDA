\chapter{Introduction}\label{ch:introduction}
This manual describes the AIDA \gls{tlu} designed for the \href{http://aida2020.web.cern.ch/}{AIDA-2020 project} by David Cussans\footnote{University of Bristol, Particle Physics group} and Paolo Baesso\footnote{University of Bristol, Particle Physics group}.\\
The unit is designed to be used in High Energy Physics beam-tests and provides a simple and flexible interface for fast timing and triggering signals at the AIDA pixel sensor beam-telescope.\\
The current version of the hardware is an evolution of the \href{https://twiki.cern.ch/twiki/bin/view/MimosaTelescope/TLU}{EUDET-TLU} and the \href{https://www.ohwr.org/projects/fmc-mtlu/wiki}{miniTLU} and is shipped in a metallic case that includes an \gls{fpga} board, the \gls{tlu} \gls{pcb} and an additional power module: the \gls{fpga} is responsible for all the logic functions of the unit, while the \gls{pcb} contains the clock chip, discriminator and interface blocks needed to communicate with other devices. The power module contains programmable \gls{dac} to power photomultipliers and \gls{led} indicators.\\
The current version of the \gls{pcb} is \brd and is designed to plug onto a carrier \gls{fpga} board like any other \gls{fmc} mezzanine board, although its form factor does not comply with the ANSI-VITA-57-1 standard.\\

\section{Overview}
The AIDA \gls{tlu} provides timing and synchronization signals to test-beam readout hardware.\\
The hardware can provide an internally generated low-jitter 40~MHz clock or can accept an external clock reference. The external reference clock frequency is not required to be 40~MHz but other values require a dedicated configuration of the clock circuitry on the board. \\
The \gls{tlu} accepts asynchronous trigger signals from up to six external sources, such as beam-scintillators, and generate synchronous signals (including global trigger or control signals) to send to up to four devices under tests. The logic function used to generate the trigger can be defined by the user among all the possible logic combinations of the inputs.\\
Depending on the chosen mode of operation, the \gls{tlu} can accept busy signals or other veto signals from \gls{dut}s and react accordingly, for instance avoiding any further trigger until all the busy signals have been de-asserted.\\
Whenever a global trigger is generated by the unit, a 48-bit time-stamp is attached to it. This time stamp is based on the 40~MHz clock. The unit records a fine-grain time stamp with 780~ps resolution for each signal involved in the trigger decision.\\
The configuration parameters and data are sent and received via the \href{https://www.ohwr.org/projects/ipbus}{IPbus}. IPbus is a simple way to control and communicate TCA-based hardware via the UDP/IP protocol.\\
The \gls{tlu} is shipped with an \gls{fpga} board already programmed with the latest version of the firmware needed to operate the unit. New features and bug fixes are continuously being implemented by the developing team.\\
The unit requires 12~V to operate. Power can be provided using the circular socket located on the back panel. See section~\ref{ch:backpanelintro} for details on compatible connectors.\\
During normal operation the current drawn by the unit is about 1~A.

\section{Front panel}\label{ch:frontpanel}
The front panel of the \gls{tlu} is shown in figure~\ref{fig:frontpanel}; from left to right, the main elements are:
\begin{itemize}
  \item \gls{sfp} cage
  \item 4 \gls{hdmi} connectors for devices under test. Each connector has a \gls{rgb} LED used to indicate the port status (see section~\ref{ch:frontpanelintro}).
  \item 1 LEMO connector for \gls{lvds} clock input/output. This is a 2-pin LEMO series 00 connector\footnote{Part number EPG.00.302.NLN. An example of mating part is LEMO FGG.00.302.CLAD35}. A \gls{rgb} \gls{led} indicator is used to signal whether the port is configured as input or output.
  \item 6 LEMO Trigger inputs. These are standard 1-pin LEMO connectors\footnote{LEMO EPK.00.250.NN. Mates with any LEMO 00.250 connector}. Each input has a \gls{rgb} \gls{led} indicator.
  \item 4 LEMO connectors to provide power to photomultipliers. This is a 4-pin connector with 9-mm diameter\footnote{LEMO part number EXP.0S.304.HLN. Mates with LEMO part FFA.0S.304.CLAC44 or similar.}. For the pin-out see section~\ref{ch:frontpanel}.
  \begin{alertinfo}{Note}
    To reduce the cost of a unit, some modules are not equipped with these connectors and the front panel holes are blanked by a plastic board.\\
    If necessary, it is possible to solder the connectors at a later stage, since all the necessary circuitry is present. This requires disassembling the unit, removing the top cover. See section~\ref{ch:inspection} for details.
  \end{alertinfo}
  \item Green \gls{led} indicators for power (+12 V) and regulators (+5 V and -5 V).
\end{itemize}
\begin{figure}
  \centering
  \includegraphics[width=.950\textwidth]{./Images/frontPanel.pdf}
  \caption{View of the TLU front panel.}
  \label{fig:frontpanel}
\end{figure}

\section{Back panel}\label{ch:backpanelintro}
The \gls{tlu} back panel is shown in figure~\ref{fig:backpanel}; from left to right, the main elements are:
\begin{itemize}
  \item RJ45 connector to communicate with the hardware using IPBus.
  \item USB-B port used to flash the internal logic with a new version of the firmware. See section\ref{ch:fpgahardware} for details.
  \begin{alertinfo}{Note}
    This port should be left disconnected if planning to use the self-boot capability of the internal logic. If a cable is detected, the \gls{fpga} will not load the pre-flashed firmware at power-up.
  \end{alertinfo}
  \item USB-B port used to communicate with the \gls{fpga} \gls{uart} port.
  \item Power connector\footnote{Switchcraft 722A; mates with a $\phi$~5.5 mm jack with $\phi$~2.1 mm central pin. For instance use Lumberg 1633 02.}. Central pin is +12 V. It is recommended to use a power supply capable of providing at least 1~A.
\end{itemize}
\begin{figure}
  \centering
  \includegraphics[width=.950\textwidth]{./Images/backPaneldoc.pdf}
  \caption{View of the TLU back panel.}
  \label{fig:backpanel}
\end{figure}

\section{FPGA and firmware}\label{ch:fpgahardware}
The firmware developed at University of Bristol is targeted to work with the Enclustra  AX3 board, which must be plugged onto a PM3 base, also produced by \href{http://www.enclustra.com/en/home/}{Enclustra}. The firmware is written on the \gls{fpga} using a \gls{jtag} interface. Typically a breakout board will be required to connect the Xilinx programming cable to the Enclustra PM3. All these components are included in the \gls{tlu} enclosure so the user can upload a new version of the firmware by simply connecting a \gls{usb}-B cable in the back panel of the unit.\\
At the time of writing this work\footnote{\monthyeardate\today} the AX3 is the only \gls{fpga} for which a firmware has been developed. However, we plan to ship future versions of the \gls{tlu} with a custom made \gls{fpga} designed by Samer Kilani.\\
Each unit is shipped with the latest version of the firmware written onto its boot loader \gls{eeprom}; at power up, the unit will automatically retrieve the firmware from the \gls{eeprom} and program itself.
\begin{alertinfo}{Note}
    If the \gls{fpga} detects a programming cable connected it will not load the firmware from its memory after a power cycle.\\
    It is recommended to leave the \gls{usb} cable disconnected from the back panel unless there is the intention to re-program the firmware.
\end{alertinfo}
The latest version of the firmware can be found on the project github repository (named \href{https://github.com/PaoloGB/firmware_AIDA}{firmware\_AIDA}).\\
The user can decide to configure the unit with a new version of the firmware that will remain active until the \gls{tlu} is powered off (standard programming). It is also possible to write the \gls{eeprom} to replace boot program with a new one (configuration memory programming). Both procedures are described below.
Programming the \gls{fpga} requires the Vivado Lab Tools\footnote{Available free \href{https://www.xilinx.com/support/download.html}{on the Xilinx website}}. Depending on the hardware installed internally, some additional drivers might be required to correctly use the \gls{jtag} cable.\\

\subsection{Standard programming}
\subsection{Configuration memory programming}

\section{Inspection}\label{ch:inspection}
At some point someone, somewhere, will want to disassemble the unit; the top cover of the unit can only slide away when either the front or back frame are removed.
\begin{alertinfo}{Note}
    Simply removing the corner screws on the panels will only allow to remove the plates but not accessing the inside of the unit.
\end{alertinfo}
The frames are held in place by 4 screws hidden behind the corner covers.\\Figure~\ref{fig:dismantle} shows the correct procedure to remove the cover: A) the easiest way to remove the cover is by removing the back frame. B) Do not remove the corner screws in the plate. C) Remove the two corner covers from the frame. They are only held in place by pressure and can be removed by hand. Once done, remove the 4 Philips screws located behind (green circles). D) unscrew the Philips screw at the bottom of the unit holding the frame in place. E) remove the frame and the back panel. Be careful to not damage the cables connecting the panel to the electronics. F) Slide the top cover away.\\
The same procedure can be repeated with the front frame, if necessary. In this case, the user must also disconnect the front panel from the electronics by removing the countersunk screws connected to the \gls{hdmi} ports and the powermodule.
\begin{figure}
\centering
    \subfloat[A]{\includegraphics[width=.45\textwidth]{./Images/View1.png}}\hfil
    \subfloat[B]{\includegraphics[width=.45\textwidth]{./Images/View6.png}}
    \subfloat[C]{\includegraphics[width=.45\textwidth]{./Images/View2.png}}\hfil
    \subfloat[D]{\includegraphics[width=.45\textwidth]{./Images/View3.png}}
    \subfloat[E]{\includegraphics[width=.45\textwidth]{./Images/View4.png}}\hfil
    \subfloat[F]{\includegraphics[width=.45\textwidth]{./Images/View5.png}}
    \caption{Steps to remove the cover from the unit.}
    \label{fig:dismantle}
\end{figure}
%\section{Preparation}
%Before powering the \gls{tlu} it is necessary to follow a few steps to ensure the board and the \gls{fpga} work correctly.\\
%
%Currently, it is recommended to use the following:
%\begin{itemize}
% \item MA-PM3-W-R5: Mars PM3 base board
% \item MA-AX3-35-1I-D8-R3: Marx AX3 module (hosts a Xilinx XC7A35T-1CSG324I )
% \item MA-PM3-ACC-BASE: Accessory kit, including a \gls{jtag} breakout board to connect Xilinx programming cables. Also includes a 12~V power supply to power the PM3.
%\end{itemize}
%
%\section{I/O voltage setting}
%The I/O pins of the PM3 can be configured to operate at 2.5~V or 3.3~V; the factory default is 2.5~V but the \brd requires 3.3~V logic. The user should make sure to select the appropriate voltage by operating on DIP-switch CFG-A/S1200 (pin 1 set to ON).\\For reference, a top view of the board is provided in the appendix at page~\pageref{ch:appendix}.\\
%\begin{alertinfo}{Warning}
%    Please double check the PM3 board manual for the correct way to change the I/O voltage setting. Enclustra has been changing their hardware recently.
%\end{alertinfo}
%
%\section{Xilinx programming cable}
%The \gls{jtag} pins on the PM3 are located on the header J800 (20-way, 2.54~mm pitch). The breakout board provided by Enclustra sits on top of the header and connects the pins to a 14-way Molex milli-grid header so that it is possible to plug the Xiling programming cable directly onto it. However, when the \brd is mounted on a base plate as shown in figure~\ref{fig:TLUplate}, the breakout board has to be detached from the PM3 because it interferes with the mounting screws.\\
%The connection between J800 and the breakout can be achieved by using two standard 20-way \gls{idc} cables as shown in figure~\ref{fig:XilinxCable}.
%\begin{figure}[h]
%  \centering
%  \includegraphics[width=.50\textwidth]{./Images/TLU_plate.jpg}
%  \caption{\brd and PM3 mounted on a base plate: in this configuration it is not possible to install the breakout board on the PM3 because the mountings screws are in the way.}\label{fig:TLUplate}
%\end{figure}
%\begin{figure}
%  \centering
%  \includegraphics[width=.80\textwidth]{./Images/XilinxCable.jpg}
%  \caption{Connecting the Xilinx programming cable to the PM3 in an ugly (but effective) way.}\label{fig:XilinxCable}
%\end{figure}
